\Chapter{Bevezetés}

A szakdolgozatom célja egy olyan program megírása, amely egy adott testet úgy módosít, hogy a testet dobókockaként használva az oldalak előfordulása az általunk megadott értékekhez közelítenek.
Azért nem bízzuk a test teljes generálását a programra, mert az egyes $n$-oldalú testekhez nem egy testháló létezik, és így több testhálóhoz is le tudjuk tesztelni a programot.

A probléma megoldása nem triviális, érdekes optimalizálási feladat.
A módosításoknál figyelnünk kell az előfordulási arányok mellett a test konvexitására, és a nem háromszög alakú oldallapok síkban maradására, hogy ne "törjön" több háromszög alakú oldalra.

Ehhez a problémakörhöz kell egy fizikai szimulációs környezetet létrehozni, ami a dobási kísérletek elvégzéséért felel, valamint a testeknek egy általános reprezentációja, amiken a módosításokat könnyen el tudjuk végezni.

Az optimalizálások során azt feltételezzük, hogy a keresett test létezik, de pontos egyezés helyett valamilyen hiba szerint fogjuk az alakzatokat elfogadni.
A dolgozatban több különböző hibaszámítást fogunk megvizsgálni, és ezeket fogjuk összehasonlítani, hogy számunkra melyik használata lenne a legelőnyösebb.

A dolgozatban részletes leírásra kerül a szimuláció felépítése, az adatok tárolása és az eloszlások közötti távolság mérése.