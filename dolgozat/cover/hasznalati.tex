\pagestyle{empty}

\noindent \textbf{\Large CD Használati útmutató}

\vskip 1cm

\noindent \textbf{A CD az alábbi jegyzékeket tartalmazza:}
\begin{itemize}
	\item\textit{dolgozat:} A szakdolgozat szövege pdf illetve LaTeX állományként.
	\item\textit{source{\_}code:} A futtatható kódok találhatóak benne.
\end{itemize}

\noindent\textbf{A \textit{source{\_}code} jegyzék tartalma:}
\begin{itemize}
	\item\textit{jelly.py:} A testek reprezentálásához szükséges osztályok találkatóak meg benne.
	\item\textit{chi2{\_}test.py:} A $\chi^2$ próba tesztelése.
	\item\textit{lambda{\_}value{\_}test.py:} A $\lambda$ érték meghatározására használt kód.
	\item\textit{mse{\_}vs{\_}chi2.py:} Az MSE és $\chi^2$ összehasonlítása legkisebb négyzetek módszerével.
	\item\textit{plane{\_}test.py:} Három pontra illeszkedő sík meghatározása.
	\item\textit{rotate{\_}test.py:} Forgatások tesztelésére szolgáló program.
	\item\textit{velocitylengthgraph.py:} A dobások során mérhető mozgásvektorok összhosszúsága által megadott grafikont kirajzoló program.
	\item\textit{tetrahedron{\_}test.py, doubletetrahedron{\_}test.py, octahedron{\_}test.py:} A tényleges optimalizáló kódok különböző testekre.
	\item\textit{graphs:} A kapott grafikonokat tartalmazó jegyzék.
	\item\textit{objects:} Az optimalizálás során elért objektumokat tartalmazza.
\end{itemize}

\noindent\textbf{A kódok futtatása:}

\textit{A programok Python $3.9.4$-ben lettek megírva.}
\textit{A futáshoz szükséges a Matplotlib, NumPy, SciPy könyvtárak telepítése.}

Az egyes kódokat parancssorból a \textit{source{\_}code} jegyzékbe való navigálás után a \textit{python} paranccsal tudjuk futtatni.
