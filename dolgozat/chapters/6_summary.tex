\Chapter{Összefoglalás}

A dolgozatban láthattunk néhány konkrét példát arra, hogy egy-egy diszkrét eloszláshoz hogyan tudunk becslést adni egy olyan testre, melynél a lapokhoz tartozó valószínűségek közelítik a célként kitűzött értékeket. Ehhez elkészült egy szimulációs környezet Python programozási nyelven, illetve különböző heurisztikák a test alakjának módosításához.

% A dolgozat célja, a módosító program elkészítése sikerült.
% Láthattuk, hogy képes alakzattól és eloszlástól függetlenül konvergálni az elvárt testhez.

Mivel egy olyan optimalizálási eljárásról van szó, melynél az aktuális test jósági értékének becslése statisztikai próbával ellenőrizhető, ezért a konvergencia vizsgálatánál nem várható el ezen értékek monotonitása az iterációk függvényében. Érdekes részproblémákat jelentett a dobássorozatok hosszának, és a megfelelő megbízhatósági értékek meghatározása.
A kapott eredmények értékeléséhez főként a konvergenciát szemléltető grafikonok, a konkrét hiba értékek és a statisztikai próbák eredményei szolgáltak.

A nagyobb problémákat az optimalizáló algoritmusban bevezetett $\lambda$ érték meghatározása, valamint a lassú futási idő jelentette.
Az utóbbin úgy lehetne javítani például, hogy a dobássorozatok elvégzését kiszervezzük \textit{C} programozási nyelven megírt modulba, melynek függvényei meghívhatók Python-ból.
Így tudnánk nagyobb mintára végezni a szimulációkat, pontosabb eredményeket kaphatnánk, illetve megvizsgálhatnánk a test eredeti alakja, az elvárt eloszlás és a $\lambda$ közötti összefüggést.

A tesztelések során a testeknek csak háromszög oldallapjuk volt, mivel a komplexebb esetekben az oldallapok módosításánál ügyelnünk kell arra, hogy háromnál több pont esetében a lapok csúcsai biztosan egy síkban maradjanak.
A felhasznált heurisztika a konvexitást megőrizte, de a nem háromszög oldallapokat "széttörte" több oldallapra.
További fejlesztés lehetne ehhez a problémához ennek a hibának a kezelése, vagy egy olyan módszer keresése, amely síkban tudja tartani az összes oldallapot.

A fent említett módosításokat elvégezve meg lehetne vizsgálni, hogy ugyan azt az eloszlást különböző testekből kiindulva hogyan tudja közelíteni a program (például kocka és dupla tetraéderre nézve).
Ezen felül lehetne vizsgálni, hogy a kapott test oldalainak a területe hogyan változik a módosítások közben.
Továbbá, valamilyen vizuális megjelenítőt lehetne készíteni, amivel láthatjuk a testet optimalizálás alatt, hogy hogyan változik.

A jelenlegi verzió az elvárt valószínűségeket mindig egy adott oldalhoz köti. Érdemes lenne úgy módosítani, hogy az optimalizálás közben rendelje hozzá az oldalakhoz az értékeket, és ezeknek a sorrendjét közben tudja változtatni.
Felmerülhet olyan probléma, hogy két oldalhoz tartozó elvárt értékeket folyamatosan cseréli, és emiatt egy ponton túl már nem tud jobb eredményeket adni az optimalizálási folyamat.
Ehhez egy külön leállási feltétel vizsgálatot kellene csinálni.

Az említett módosítások, bővítések nélkül is az elvárt feltételeknek megfelelően működik a program, és kellő futási idő után elég pontos eredményeket tud adni.
Az így kapott kockák elkészítése a WaveFront OBJ fájlokba történő mentése után 3D nyomtatók segítségével (természetesen a megfelelő beállítások használatával) könnyen előállíthatóak.
