\Chapter{Tervezés}

A dobótest felépítése

Csak a konvex esetekkel foglalkozunk, így elegendő a testnek csak a csúcspontjait eltárolni.
A test szilárdságát azzal biztosítjuk, hogy minden csúcspont közé egy rugót teszünk, amely a visszapattanás során eldeformálódott ponthalmazt az eredeti alakjába próbál visszahúzni.
A rúgóállandótól függően lesz az objektumnak "zselé" fizikája, amely biztosítja a visszapattanást.
Minden csúcspontnak külön számoljuk a rá ható erőket (gravitációs erő, rugóerők), valamint a sebességét.
A zselatinossága miatt akkor tekintjük a testet nyugalmi állapotban amikor a pontok sebességvektorainak öszhosszúsága egy küszöbérték alá esik. (mindig fog pattogni egy kicsit a talajon)
A dobott értékek meghatározásához a talajon elhelyezkedő pontokat fogjuk figyelembe venni, mivel ha a test a dobás után megáll mindig egy oldallapján fog megállni. (tetraédernél a dobott értéket a felül lévő csúcsa jelöli, míg a kockánál egy oldallap. mindkét esetben a test a talajjal egy teljes oldallapján érintkezik)
A nyugalmi állapothoz hasonlóan azokat a pontokat tekintjük talajon lévőnek, amelyeknek a $z$ koordinátájuk $0$ és a megadott küszöbérték között található.


A dobás szimulálása

A fizikai szimulálás minden iterációja a következő lépésekből ál:
1. a pontokra ható eredő erők meghatározása
(gravitációs erő, rugóerők, sebességvektor)
2. a talajon/talaj alatt lévő lévő pontok pattanásának kezelése
(előfordul, hogy egy iteráció után egy pont a $z=0$ sík alá esik, ilyenkor a pontot a $z=0$ síkra helyezzük)
(pattanás során az eredő erő z komponensét kinullázzuk)
3. mozgásvektorok frissítése az eredő erők alapján
4. a pontok mozgatása a frissített mozgásvektoroknak megfelelően

A dobások szimulálásához a testet mindig egy random pozícióba forgatjuk, és abből a pozícióbol ejtjük le.
Ez a random forgatás nem ekvivalens azzal, hogy a testet random elforgatjuk az $x$, $y$ és a $z$ tengely körül.


Random forgatás

A tényleges random forgatást az alábbi módon érhetjük el:
A testet eltoljuk úgy, hogy a középpontja a koordináta rendszerünk origójában legyen. (a test középpontját a csúcsok súlypontjával közelítjük)
Az egyik csúcspontot kiválasztjuk pivotelemnek.
Választunk egy pontot az origó középpontú origó-pivot sugarú gömbfelületen.
A testet úgy forgatjuk $x$, $y$ és $z$ szerint, hogy a pivot csúcs a véletlenszerűen választott pontunka essen.
Majd a testet elforgatjuk az origón és pivotcsúcson átmenő egyenes körül egy $0^\circ \leq \alpha < 360^\circ$ szöggel.

Az utolsó lépést csak úgy tudjuk megvalósítani, hogy a testet a random pontba forgatás után úgy forgatjuk hogy a pivot az egyik tengelyre illeszkedjen, majd a testet elforgatjuk alpha szöggel az a tengely körül amelyikre illeszkedik a pivot csúcs. Ez után csak annyi a teendőnk hogy a testet visszaforgatjuk abba a pozícióba, hogy a pivot illeszkedjen a választott pontunkra.

Észre lehet venni, hogy a random pontba forgatás, egyik tengelyre való forgatás, majd ismét a random pontba forgatás szekvenciája helyettesíthető azzal, ha egyből az egyik tengelyre forgatjuk rá a pontot, majd a tengely körüli forgatás  után mozgatjuk végleges pozíciójába.

Az algoritmusunk így a következő lépésekből fog állni:

1. eltoljuk a testet úgy, hogy az origó legyen a középpontja

2. kiválasztjuk az egyik csúcsot pivotnak

3. generálunk egy random pontot az origó középpontú origó-pivot távolsággal megegyező sugarú gömbfelületen

4. a testet elforgatjuk $z$ majd $y$ szerint úgy, hogy a pivot csúcs illeszkedjen az $x$ tengelyre

5. elforgatjuk a testet egy $0^\circ \leq \alpha < 360^\circ$ szöggel az $x$ tengely körül

6. a testet $y$ és $z$ szerinti forgatásokkal úgy mozgatjuk, hogy a pivot csúcs illeszkedjen a választott pontra

7. visszatoljuk a testet az eredeti "helyére" (súlypont alapján)

(megjegyzés: a test eltolásait kivehetjük külön függvénybe)

Metódusok a test módosítására

1. random pont mozgatása

Ez a módszer a test egy véletlenszerűen választott pontját egy random vektorral eltolja.
Ha az így kapott test az $N$ dobási kísérlet után nagyobb megbízhatósági szinttel rendelkezik, mint az eredeti test, akkor a következő iterációban ebből a testből indulunk ki, egyébként elvetjük, és előlről kezdjük az algoritmust.
könnyen implementálható, de a futási eredménye nagyban függ a generált számoktól, valamint kis dobási minta esetén hamarabb kapunk fals eredményt mint a többi metódusban.

2. oldallapok méretének növelése/csökkentése

A test statisztikájának meghatározása után az oldallapokat módosítjuk annak függvényében hogy az adott lap gyakorisága az elvárt érték alatt vagy felett helyezkedik el.
A módosítás a lap síkjában történik, és minden csúcspontot az oldallap súlypontjához képest mozdítjuk el. (ábra!)
Mindig egység hosszúságú vektorokkal mozdítjuk el a csúcspontokat.
A random pont mozgatásos módszerhez képest jelentősebben gyorsabb, de kis mintájú dobások esetén a kapott értékek nem a valóságot tükrözik. (egy $N=200$ darabszámú dobássorozat esetén kapott $p=0.8$ szignifikancia szint nem jelenti azt, hogy $N=1000$ esetén is hasonló lesz a $p$ érték)

3. oldallapok méretének növelése/csökkentése a kapott és várt statisztikák függvényében

Nagyon hasonlít az előzőhöz, csupán annyiban különbözik az előzőtől, hogy amikor az oldallapok pontjait mozgatjuk a vektorok hossza arányos a kapott és várt gyakoriságok különbségével.



\Section{Táblázatok}

Táblázatokhoz a \texttt{table} környezetet ajánlott használni.
Erre egy minta \aref{tab:minta}. táblázat.
A hivatkozáshoz az egyedi \texttt{label} értéke konvenció szerint \texttt{tab:} prefixszel kezdődik.

\begin{table}[h]
\centering
\caption{Minta táblázat. A táblázat felirata a táblázat felett kell legyen!}
\label{tab:minta}
\begin{tabular}{l|c|c|}
a & b & c \\
\hline
1 & 2 & 3 \\
4 & 5 & 6 \\
\hline
\end{tabular}
\end{table}

\Section{Ábrák}

Ábrákat a \texttt{figure} környezettel lehet használni.
A használatára egy példa \aref{fig:cimer}. ábrán látható.
Az \texttt{includegraphics} parancsba 
Az ábrák felirata az ábra alatt kell legyen.
Az ábrák hivatkozásához használt nevet konvenció szerint \texttt{fig:}-el célszerű kezdeni.

\begin{figure}[h]
\centering
\includegraphics[scale=0.3]{images/me_logo.png}
\caption{A Miskolci Egyetem címere.}
\label{fig:cimer}
\end{figure}

\Section{További környezetek}

A matematikai témájú dolgozatokban szükség lehet tételek és bizonyításaik megadására.
Ehhez szintén vannak készen elérhető környezetek.

\begin{definition}
Ez egy definíció
\end{definition}

\begin{lemma}
Ez egy lemma
\end{lemma}

\begin{theorem}
Ez egy tétel
\end{theorem}

\begin{proof}
Ez egy bizonyítás
\end{proof}

\begin{corollary}
Ez egy tétel
\end{corollary}

\begin{remark}
Ez egy megjegyzés
\end{remark}

\begin{example}
Ez egy példa
\end{example}
