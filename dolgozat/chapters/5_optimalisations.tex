\Chapter{Testek optimalizálása}

Érdemes megnézni, hogy egyes testeknél hogyan tudunk konvergálni különböző eloszlásokhoz.
Ehhez a \ref{sect:ratiomodification} szakaszban bemutatott heurisztikát fogjuk használni.
Addig módosítjuk a testeket, amíg az oldallapokhoz tartozó relatív hiba az általunk megadott küszöbérték alá nem esik.
Ezt a küszöböt $\chi^2$-re nézve az alábbi összefüggéssel írhatjuk fel:
\[
\chi^2 < N \cdot error^2,
\]
ahol $N$ a dobássorozat száma és $error$ az általunk megadott hibaküszöb. 

\Section{Tetraéder}

[0.1, 0.2, 0.3, 0.4]

Dark Souls Orange Dice
[1/6, 1/3, 1/3, 1/6]

\Section{Dupla tetraéder}

[1/12, 2/12, 3/12, 3/12, 2/12, 1/12]

[2/15, 2/15, 2/15, 2/15, 2/15, 1/3]

\Section{Oktaéder}

sum of 7 coin
[1/128, 7/128, 21/128, 35/128, 35/128, 21/128, 7/128, 1/128]

2 szemben lévő oldal nagyobb
[2/8, 2/8, 1/12, 1/12, 1/12, 1/12, 1/12, 1/12]
