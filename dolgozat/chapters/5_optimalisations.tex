\Chapter{Testek optimalizálása}
\label{ch:opt}

Érdemes megnézni, hogy egyes testeknél hogyan tudunk konvergálni különböző eloszlásokhoz.
Ehhez a \ref{sect:ratiomodification} szakaszban bemutatott heurisztikát fogjuk használni.
Addig módosítjuk a testeket, amíg az oldallapokhoz tartozó relatív hiba az általunk megadott küszöbérték alá nem esik.
Ezt a küszöböt $\chi^2$-re nézve az alábbi összefüggéssel írhatjuk fel:
\[
\begin{array}{c}
\chi^2=\sum\limits_{i=1}^{n} \frac{(e_i\cdot (1+error) - e_i)^2}{e_i} = \sum\limits_{i=1}^{n} \frac{(e_i + e_i\cdot error - e_i)^2}{e_i} = \sum\limits_{i=1}^{n} \frac{(e_i\cdot error)^2}{e_i} = \\
= \sum\limits_{i=1}^{n} \frac{e_i^2\cdot error^2}{e_i} = \sum\limits_{i=1}^{n}{e_i\cdot error^2} = error^2 \sum\limits_{i=1}^{n} e_i = error^2 \cdot N,
\end{array}
\]
ahol $e_i$ az oldalakhoz tartozó elvárt gyakoriság, $N$ a dobássorozat száma és $error$ az általunk megadott hibaküszöb.
Ezt a küszöböt tudnánk igazítani ahhoz is, hogy a módosított testet milyen szignifikancia szinten szeretnénk elfogadni.
Testenként változó, hogy egy adott $\alpha$ szignifikancia szinthez tartozó  $\chi^2$ érték maximum mennyi lehet, és annak nagysága nagyban függ a dobássorozatok méretétől, ezért a fent leírt képlettel fogjuk számolni a hibaküszöböt.

A fejezet további részében az optimalizálások kerülnek bemutatásra.
Mindegyik mérésről lesz egy leírás, hogy miért azt az eloszlást választottuk, milyen eredményre számítunk.
A testek és a várt eloszlás leírása után fognak szerepelni a futásnál használt paraméterek és a konvergencia görbe.
Végül az egyéb tapasztalatok jönnek, hogy mennyire hasonlít a test az általunk elképzelt alakzatra.

\Section{Tetraéder}

Az alap tetraéder, amit módosítani fogunk 50 egység hosszú oldallapokkal rendelkezik.
Azért, hogy a módosított alakzatokat $\alpha = 0.05$ szignifikancia szinten el tudjuk fogadni a $\chi^2 < 7.815$ feltételnek teljesülnie kell.

A módosító algoritmusok leírásánál ezen a testen végeztünk teszteket, és mindig a $[0.1, 0.2, 0.3, 0.4]$ eloszlást közelítettük.
Azért ezeket az értékeket választottuk, mert itt egyértelműen látszódik az egyes oldallapok közötti különbség, valamint könnyen meg tudjuk határozni ránézésre, hogy a mért frekvenciák mennyire térnek el az elvárt értékektől.

\begin{figure}[h!]
	\centering
	\includegraphics[scale=0.7]{images/tetrahedron_01.png}
	\caption{Az teszt eloszláshoz tartozó konvergencia görbe.}
	\label{fig:tetra01}
\end{figure}
%end chi2 = 1.9608333333333332

\begin{figure}[h!]
	\centering
	\includegraphics[width=\textwidth]{images/tetra01obj.png}
	\caption{A teszt eloszláshoz kapott alakzat.}
	\label{fig:tetra01obj}
\end{figure}

\Aref{fig:tetra01} grafikonon látható a konvergenciája a testnek az említett eloszláshoz.
Látható, hogy a választott teszt eloszlás nem csak vizsgálat szempontjából előnyös, hanem könnyen elérhető.
Az optimalizálással kapott testről láthatunk néhány képet különböző szögekből \aref{fig:tetra01obj} ábrán.

A következő test a \textit{Dark Souls: The Board Game}-ben \cite{darksouls} használt narancssárga dobókocka közelítése.
A kocka oldallapjain található értékek a következőek: 1 darab egyes, 2 darab kettes, 2 darab hármas és 1 darab négyes.
Mivel csak négy érték fordul elő, ezért lehet a testet egy tetraéderrel közelíteni, aminek várhatóan kettő oldallapja nagyobb, kettő oldallapja pedig kisebb lesz.
Így a közelítendő eloszlás:
$$
\left[\frac{1}{6}, \frac{1}{3}, \frac{1}{3}, \frac{1}{6}\right].
$$

\begin{figure}[h!]
	\centering
	\includegraphics[scale=0.7]{images/tetrahedron_02.png}
	\caption{Az második tetraéder konvergencia görbéje.}
	\label{fig:tetra02}
\end{figure}
%end chi2 = 3.689

\begin{figure}[h!]
	\centering
	\includegraphics[width=\textwidth]{images/tetra02obj.png}
	\caption{A \textit{Dark Souls} társasjáték narancssárga dobókockájának a közelítése.}
	\label{fig:tetra02obj}
\end{figure}

\Aref{fig:tetra02} grafikonon látható, hogy a választott eloszlás elérése nem olyan triviális.
Az algoritmus követ el hibákat, amelyek a test alakja miatt nem okoznak látványos változást a görbén.

A kapott test oldallapjai egyenlő szárú háromszögekhez hasonlítanak, amelyeknek az alapjuk más hosszúságú.
Felismerhető, hogy kis hibával különböznek az egyforma lappárok, valamint az alap feltételezésünk, hogy a ritkább előfordulású oldallapok területe kisebb lesz beigazolódott.

\newpage

\Section{Dupla tetraéder}

A dupla tetraéder a sima változatához hasonlóan $50$ egység hosszú oldalélekkel fog rendelkezni.
A $\chi^2$ próbánál, hogy $\alpha = 0.05$ szignifikancia szinten el tudjuk fogadni a módosítást a $\chi^2$-nek kisebbnek kell lennie mint $11.07$.

A tetraéderhez hasonlóan egy egyszerűbbnek tűnő eloszlással fogunk kipróbálni.
Az eloszlás:
\[
\left[\frac{1}{12}, \frac{2}{12}, \frac{3}{12}, \frac{3}{12}, \frac{2}{12}, \frac{1}{12}\right].
\]

\begin{figure}[h!]
	\centering
	\includegraphics[scale=0.7]{images/doubletetrahedron_01.png}
	\caption{Az első dupla tetraéder konvergencia görbéje.}
	\label{fig:doubletetra01}
\end{figure}
%after test chi2 = 1.9919999999999995

\begin{figure}[h!]
	\centering
	\includegraphics[width=\textwidth]{images/double01obj.png}
	\caption{Az első dupla tetraéder.}
	\label{fig:double01obj}
\end{figure}

\Aref{fig:doubletetra01} grafikonon látható, hogy az 5. iterációtól kezdve elkezdett az algoritmus egyre rosszabb értékeket kapni, és csak utána tudott a megfelelő értékhez konvergálni.
Ez amiatt történt, hogy a $\chi^2$ érték nagyon közel került a küszöbértékhez, és a következő iterációkban lévő fokozatos romlás a $\lambda$ értéknek köszönhető.
Ilyen esetekre számítottunk amikor \aref{sect:ratiomodification} szakaszban a $\lambda$ értéket akartuk meghatározni.

A következő módosításnál a
\[
\left[\frac{2}{15}, \frac{2}{15}, \frac{2}{15}, \frac{2}{15}, \frac{2}{15}, \frac{1}{3}\right]
\]
eloszlást szeretnénk elérni.
Arra számítunk, hogy egy olyan "cinkelt" testet kapunk, amelyen az egyik oldal az esetek harmadában lesz a dobások eredménye, a maradék öt oldalnak pedig az előfordulása egyenletes lesz a fennmaradó valószínűségre nézve.

\begin{figure}[h!]
	\centering
	\includegraphics[scale=0.7]{images/doubletetrahedron_02.png}
	\caption{A második dupla tetraéder konvergencia görbéje.}
	\label{fig:doubletetra02}
\end{figure}
%after test chi2 = 9.489499999999996

\begin{figure}[h!]
	\centering
	\includegraphics[width=\textwidth]{images/double02obj.png}
	\caption{A második dupla tetraéder.}
	\label{fig:double02obj}
\end{figure}

\Aref{fig:doubletetra02} grafikon az előzővel ellentétben nem mutatott extrém eltéréseket, de sokkal nehezebben tud konvergálni az elvárt értékekhez.
A kapott testen nem olyan látványos az oldallapok közötti különbség mint amire számítottunk.

\newpage

\Section{Oktaéder}

A kezdeti oktaéder csúcsai a középpontjától 50 egység távolságra helyezkednek el, így az oldalélek kicsivel nagyobbak az előzőleg tesztelt testekénél.
A könnyebb inicializálás miatt választottuk a méretének ezen meghatározását.
Az ellenőrzésnél a $\chi^2 < 14.067$ vizsgálatnak kell teljesülnie.

\begin{remark}
A szimulációkban nagyobb hibaküszöböt használunk, mert a test alakja miatt nehezebben közelíthető az eloszlás.
A mintaszámot is növelhetnénk, de akkor a futási idő exponenciálisan nőne.
\end{remark}

Az első testtel a hét érme feldobása során kapott értékek összegét szeretnénk közelíteni.
(Az érmedobás eredményét nullának vagy egynek vesszük.)
A várt binomiális eloszlás: 
\[
\left[\frac{1}{128}, \frac{7}{128}, \frac{21}{128}, \frac{35}{128}, \frac{35}{128}, \frac{21}{128}, \frac{7}{128}, \frac{1}{128}\right].
\]

\begin{figure}[h!]
	\centering
	\includegraphics[scale=0.7]{images/octahedron_01.png}
	\caption{Az első oktaéder konvergencia görbéje.}
	\label{fig:octa01}
\end{figure}
%after test chi2 = 65.54
% wrong

\begin{figure}[h!]
	\centering
	\includegraphics[width=\textwidth]{images/octa01obj.png}
	\caption{Az első oktaéder.}
	\label{fig:octa01obj}
\end{figure}

\Aref{fig:octa01} grafikonon látható konvergencia alapján ez a legjobb eredmény amit a mérések alatt kaptunk.
Ezzel ellentétben az algoritmus leállása után végzett $\chi^2$ próbán nem felelt meg nekünk a test.
Ez annak a következménye, hogy az elvárt eloszlás nehezen közelíthetősége miatt feljebb vittük a hibaküszöböt, és így pontatlanabb mérést kaptunk.
A Python kódok lassú futási ideje miatt nem a mintavételt növeltük.

Most vizsgáljunk meg egy másik eloszlást, szintén oktaéder esetében. Mivel az egyes valószínűségek rögzítve vannak az oldallapokhoz, így meg tudjuk adni azt, hogy két szemben lévő lapnak nagyobb legyen az előfordulása.
Arra számítunk hogy a kapott testnek az alakja kicsit hasonlítani fog egy hasábra.
A várt eloszlás:
\[
\left[\frac{1}{4}, \frac{1}{12}, \frac{1}{12}, \frac{1}{12}, \frac{1}{12}, \frac{1}{12}, \frac{1}{12}, \frac{1}{4}\right].
\]

\begin{figure}[h!]
	\centering
	\includegraphics[scale=0.7]{images/octahedron_02.png}
	\caption{A második oktaéder konvergencia görbéje.}
	\label{fig:octa02}
\end{figure}
%after test chi2 = 7.92

\begin{figure}[h!]
	\centering
	\includegraphics[width=\textwidth]{images/octa02obj.png}
	\caption{Az második oktaéder.}
	\label{fig:octa02obj}
\end{figure}

\Aref{fig:octa02} grafikonról leolvasható, hogy egyszerűbben el tudunk jutni ehhez az eloszláshoz, mint az előtte választotthoz.
Ez itt azt bizonyítja, hogy maga az eloszlás is nagy szerepet játszik a konvergencia gyorsaságában.
A kapott testen látható (\ref{fig:octa02obj} ábra), hogy nem kaptunk annyira drasztikusan eltérő testet, mint vártuk.
A későbbiekben érdemes lehet egy szélsőségesebb eloszlást kipróbálni.