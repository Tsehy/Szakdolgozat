\Chapter{Összefoglalás}

A dolgozat célja, a módosító program elkészítése sikerült.
Láthattuk, hogy képes alakzattól és eloszlástól függetlenül konvergálni az elvárt testhez.

Nagyobb problémát a $\lambda$ érték, valamint a lassú futási idő okozta.
Az utóbbin úgy lehetne javítani, hogy a dobássorozatok elvégzését kiszervezzük \textit{C} programokba, ezzel gyorsítva a futást.
Így tudnánk nagyobb mintára végezni a szimulációkat, pontosabb eredményeket kaphatnánk, illetve megvizsgálhatnánk a test eredeti alakja, az elvárt eloszlás és a $\lambda$ közötti összefüggést.

A tesztelések során a testeknek csak háromszög oldallapjuk volt, mivel a komplexebb oldallapok módosításánál ügyelnünk kell arra, hogy sík maradjon minden oldal.
A használt heurisztika a konvexitást megőrizte, de a nem háromszög oldallapokat "széttörte" több oldallapra.
További fejlesztés lehetne ehhez a problémához ennek a hibának a kezelése, vagy egy olyan módszer keresése, amely síkban tudja tartani az összes oldallapot.

A fent említett módosításokat elvégezve meg lehetne vizsgálni, hogy ugyan azt az eloszlást különböző testekből kiindulva hogyan tudja közelíteni a program (például kocka és dupla tetraéderre nézve).
Ezen felül lehetne vizsgálni, hogy a kapott test oldalainak a területe hogyan változik a módosítások közben.
Esetleg valamilyen vizuális megjelenítőt lehetne készíteni, amivel láthatjuk a testet optimalizálás alatt, hogy hogyan változik.

A jelenlegi verzió az elvárt valószínűségeket mindig egy adott oldalhoz köti, érdemes lenne úgy módosítani, hogy optimalizálás közben rendeli hozzá az oldalakhoz az értékeket, és ezeknek a sorrendjét közben tudja változtatni.
Ilyenkor felmerülhet olyan probléma, hogy két oldalhoz tartozó elvárt értékeket folyamatosan cseréli, és emiatt sosem tud leállni az optimalizálás.
Ehhez egy külön leállási feltétel vizsgálatot kellene csinálni.

Az említett módosítások nélkül is az elvárt feltételeknek megfelelően működik a program, és kellő futási idő után elég pontos eredményeket tud adni.
Az így kapott kockák elkészítése a WaveFront OBJ fájlokba történő mentése után 3D nyomtatók segítségével (megfelelő beállítások használatával) könnyen előállíthatóak.